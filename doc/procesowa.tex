% By zmienic jezyk na angielski/polski, dodaj opcje do klasy english lub polish
\documentclass[polish, 12pt]{aghthesis}
\usepackage{babel}
\usepackage[utf8]{inputenc}
\usepackage{url}
\usepackage{indentfirst}
\usepackage{amsmath}
\usepackage{comment}

\author{Dawid Romanowski, Wojciech Czarny}

\title{Implementacja metody SPH \\ na procesory graficzne}

\supervisor{dr hab.\ inż.\ Krzysztof Boryczko, prof.\ nadzw.\ AGH}

\date{2014}

% Szablon przystosowany jest do druku dwustronnego, 
\begin{document}
\raggedbottom
\maketitle{}

\tableofcontents
\clearpage

\section{Wizja projektu}


	\subsection{Opis problemu}
	
	Symulacja zjawisk fizycznych przy użyciu komputerów możliwa jest jedynie z taką dokładnością na jaką pozwala nam moc obliczeniowa maszyny. W wielu przypadkach jednak dokładność numeryczna nie jest kluczowym kryterium, za to tym co tak naprawdę chce się uzyskać jest dokładność wizualna. Dlatego też przy wizualizacji symulacji modeli fizycznych stosuje się uproszczenia mające na celu naśladowanie poprawnych wyników pod względem wyglądu. W przypadku symulacji zachowania płynów nie jest możliwa wizualizacja wyników w czasie wykonywania obliczeń, gdy obliczenia uwzględniają wszystkie oddziaływania pomiędzy elementami modelu fizycznego. Konieczna jest zatem dyskretyzacja problemu oraz przyjęcie uproszczonych zjawisk lub zastąpienie ich prostszymi rozwiązaniami aproksymującymi ich zachowania. Takim podejściem jest właśnie metoda SPH, będąca uproszczeniem praw hydrodynamiki opisanym przez równania Naviera-Stokesa.
	
	$\,$
	
	Pomimo iż metoda ta została pierwszy raz zaproponowana w roku 1977, to dopiero od niedawna możliwe jest jej wykorzystanie do zaprezentowania wyników w czasie rzeczywistym. Zwłaszcza gdy procesory przestały znacząco zwiększać swoją prędkość taktowania, a zaczęła rosnąć ich koegzystentna ilość, coraz popularniejsze staje się przeprowadzanie niezależnych, zrównoleglonych obliczeń w celu pełnego wykorzystania potencjału sprzętu. Sprawdzenie wyników realizacji obliczeń metody SPH w ten właśnie sposób jest jednym z głównych założeń naszej pracy.
	
	\subsection{Cele projektu}
	
	Bardzo ciężko jest znaleźć w internecie działającą w czasie rzeczywistym implementację metody SPH wraz z wizualizacją. W technologii OpenCL, na którą się zdecydowaliśmy, taka implementacja (działająca) po prostu nie istnieje (lub nam nie udało się jej znaleźć). Rozwój algorytmów opartych o GPGPU jest rzeczą widoczną od stosunkowo niedawna, więc projekt ten jest pracą na swój sposób pionierską mającą pokazać zasadność skorzystania z tej techniki w symulacjach.
		
	Po konsultacji z opiekunem główne założenia projektu zostały określone następująco:
	
	\begin{itemize}
	
		\item Przeprowadzenie wszystkich obliczeń wymaganych do ostatecznego zasymulowania zachowania płynu metodą SPH na wielordzeniowej architekturze przy użyciu zestawu narzędzi OpenCL lub CUDA
				
		\item Optymalizacja wydajności użytych rozwiązań w celu uzyskania możliwie najpłynniejszej animacji uzyskanych wyników (przyjęto, że $\sim 30$ klatek na sekundę jest wystarczającym osiągnięciem)
		
		\item Graficzna prezentacja wyników w czasie rzeczywistym
		
	\end{itemize}
	
	
	\subsection{Opis użytkownika}
	
	Realizacja pracy zakłada istnienie dwóch rodzajów użytkowników.
	
	Pierwszym i najważniejszym jest osoba chcąca osiągnąć te same cele, których realizacji podjęli się autorzy pracy. Dla tego użytkownika zaimplementowany system oraz dokumentacja projektu są zbiorem przemyśleń i wskazówek do odtworzenia i rozwoju powstałej technologii.
	
	Drugim typem użytkownika jest osoba korzystająca z zaimplementowanego systemu w celu zobaczenia jego możliwości oraz lepszego zrozumienia metody hydrodynamiki cząstek wygładzanych poprzez manipulację parametrów układu przy użyciu interfejsu użytkownika, oraz obserwację wpływu dokonywanych zmian dzięki modułowi graficznemu aplikacji.
	
	\subsection{Początkowy opis wymagań}
	
		\subsubsection{Wymagania funkcjonalne}
		
		\begin{itemize}
		
			\item Moduł obliczeniowy
			
			\begin{itemize}
			
				\item Realizacja obliczeń związanych z symulacją cząstek płynu w skali makro równolegle na procesorach wielordzeniowych przy użyciu OpenCL
				\item Użycie trybu współdzielenia pamięci (interop) pomiędzy OpenCL i OpenGL w celu pozbycia się konieczności odczytu dużych ilości danych z pamięci procesora graficznego.
			
			\end{itemize}						
			
			\item Moduł graficzny
			
			\begin{itemize}
			
				\item Możliwość przejrzystego zaprezentowania pudła obliczeniowego (pudło oraz jego zawartość muszą być obserwowalne w trzech wymiarach przestrzennych)
				\item Możliwość poruszania kamerą (obracanie i przemieszczanie w przestrzeni trójwymiarowej)
				\item Wyświetlanie obecnej konfiguracji modelu (parametrów obliczeniowych) oraz informacji pomocniczych wyliczanych w trakcie wykonywania obliczeń poprzez graficzny interfejs użytkownika
				\item Wyświetlanie obszarów płynu w postaci sfer w możliwie najoptymalniejszy sposób, by duża ilość cząstek nie powodowała dużego narzutu obliczeniowego związanego z wyświetlaniem dużej ilości poligonów
				\item Możliwość dostosowywania wartości parametrów modelu w trakcie działania aplikacji poprzez wejście z klawiatury w celu dobrania najbardziej realistycznie wyglądającej konfiguracji
				\item Osiągnięcie płynnej animacji wyników obliczeń na poziomie przynajmniej $\sim30$ klatek na sekundę
			
			\end{itemize}	
		
		\end{itemize}
		
		\subsubsection{Wymagania niefunkcjonalne}
		
		\begin{itemize}
		
			\item Implementacja projektu w środowisku wieloplatformowym (rozwijanie równolegle w systemach Windows 8.1 oraz Linux Mint) z wykorzystaniem CMake.
			\item Zrównoleglenie możliwie jak największej ilości obliczeń
			\item Stworzenie dokumentacji, która będzie zawierała nasze przemyślenia odnośnie tworzonego produktu w celu umożliwienia potencjalnemu użytkownikowi rozwoju oprogramowania w przyszłości.
		
		\end{itemize}

\section{Przebieg pracy}
	
	\subsection{Kamienie milowe}
	
	\begin{description}
	
  		\item[Milestone 1] \hfill \\ Na samym początku, przez rozpoczęciem zapoznawania się z tematem pracy, nie mieliśmy żadnego doświadczenia i wiedzy na temat wytwarzania oprogramowania w środowisku zrównoleglonym, czy też na temat zasady działania metody hydrodynamiki cząstek wygładzonych. Jedynym w miarę znanym nam aspektem było niskopoziomowe programowanie modułów graficznych przy użyciu bibliotek OpenGL/DirectX, dlatego też w celu posunięcia prac do przodu postanowiliśmy zacząć od stworzenia prototypu części graficznej projektu.
  		
  		\item[Milestone 2] \hfill \\ Po stworzeniu działającego prototypu modułu graficznego na jego podstawie zaczęła powstawać wersja finalna struktury aplikacji. Zaczęliśmy też tworzyć prototyp modułu obliczeniowego, jako że nie mieliśmy wcześniej styczności z interfejsem programistycznym OpenCL, czy też językiem OpenCL C pierwsze przedsięwzięcia miały na celu nauczyć nas posługiwania się tą technologią i nie miały wiele wspólnego z realizacją obliczeń metody SPH, dlatego też zostały później porzucone. Gdy już nabraliśmy więcej pewności w programowaniu przy użyciu OpenCL zaczęliśmy planować podział poszczególnych etapów obliczeń pomiędzy funkcje kerneli, a następnie ich implementację. Na tym etapie zaczęły też pojawiać się rozwiązania typu użycie trybu współdzielenia pamięci na karcie graficznej oraz użycie czterowierzchołkowych płaszczyzn z obrazem sfery zwróconych w stronę kamery zamiast dużo bardziej skomplikowanych modeli sfer.
  		
 		\item[Milestone 3] \hfill \\ W trakcie ostatniego kamienia milowego dokończona została implementacja prototypu modułu obliczeniowego. Następnie, jako że do jego poprawnego działania potrzebny jest odpowiedni dobór stałych parametryzujących obliczenia, potrzebna była możliwość zmiany tych parametrów w trakcie przeprowadzania symulacji, by móc obserwować ich poprawność. W tym celu należało zaimplementować funkcjonalność graficznego interfejsu użytkownika oraz możliwość korygowania parametrów poprzez wejście z klawiatury. Na koniec pozostały ewentualne optymalizacje związane z wykonywaniem obliczeń, celem których było wykonywanie jak największej ilości obliczeń równolegle zamiast sekwencyjnie.
  			
	\end{description}
	
	\subsection{Minutki ze spotkań}
	
		\subsubsection*{11.03.2014 - Spotkanie z prowadzącym pracownię projektową (inicjujące)}
		
			\begin{itemize}
			
				\item ogólny plan spotkań oraz określenie milestone'ów (3-4 milestone'y)
				\item spotkania będą odbywać się co dwa tygodnie we wtorki
				\item przeplatane spotkania z klientem i prowadzącym PP
				\item na spotkaniach wystarczy obecność jednej osoby zaangażowanej w tworzenie projektu na raz
				\item pierwszy milestone na prototyp, zapoznanie się z technologią, na zakończenie prezentacja wyników dotychczasowej pracy
				\item dokumentacja powinna być rozwijana wraz z implementacją
				\item do opracowania krótka wizja projektu, ma być umieszczona w Confluence
						
			\end{itemize}
		
		\subsubsection*{20.03.2014 - Spotkanie z klientem (inicjujące)}
		
			Na pierwszym spotkaniu z klientem została nam zaprezentowany ogólny zarys problemu:
				
			\begin{itemize}
			
				\item klient jest już w posiadaniu aplikacji symulującej cząstki metodą SPH, jednak problem jest rozwiązywany sekwencyjnie i jedynie generuje wyniki do post-renderowania
				\item całość obliczeń zamyka się w czterech etapach:
				\begin{itemize}
				
					\item obliczenie gęstości dla każdego obszaru
					\item obliczenie różnicy ciśnień dla każdego obszaru
					\item obliczenie przyspieszenia dla każdego obszaru
					\item interpolacja prędkości i położenia w kroku czasowym dla każdego obszaru
				
				\end{itemize}
				\item problemem jest to, że liczenie oddziaływań dla każdej cząstki na podstawie wszystkich pozostałych jest zbyt kosztowne, dlatego też głównym założeniem metody SPH jest wprowadzenie promienia wygładzającego, wszystkie cząstki znajdujące się dalej niż promień wygładzania mają zerowy wpływ na rozpatrywaną cząstkę
				\item żeby nie sprawdzać odległości wszystkich cząstek wybiera się tylko $k$ najbliższych sąsiadów, wyznaczanych przy użyciu algorytmu heurystycznego
				\item dla przestrzeni jedno, dwu i trójwymiarowych wyznaczono optymalną ilość sąsiadów potrzebnych do uzyskania możliwie najrzeczywistszych rezultatów, w naszym przypadku (przestrzeń trójwymiarowa) jest to 55
			
			\end{itemize}
		
		\subsubsection*{08.04.2014 - Spotkanie z prowadzącym pracownię projektową}
		
			W trakcie spotkania przedstawiliśmy wizję projektu oraz wstępne decyzje jakie podjęliśmy w sprawie technologii jakich zamierzamy użyć w projekcie.
			
			\begin{itemize}
			
				\item przedstawiliśmy na czym polega metoda SPH
				\item krótkie uwagi odnośnie wizji projektu, ukierunkowanie na tematy, które powinniśmy dogłębniej zbadać
					\begin{itemize}
						\item powinniśmy jak najszybciej stworzyć działający prototyp, aby mając odniesienie do całości problematyki projektu móc zacząć tworzyć projekt finalny
						\item jak wyobrażamy sobie strukturę projektu
						\item co jest najistotniejsze przy realizacji, co można odłożyć na później, co nie jest niezbędne
					\end{itemize}
				\item powody dla których wybraliśmy OpenCL zamiast CUDA
					\begin{itemize}
						
						\item szersze spektrum sprzętu na którym możliwy jest rozwój oprogramowania
						\item mniejsze zaplecze gotowych rozwiązań, przez co większy stopień pionierskości projektu
						
					\end{itemize}
			
			\end{itemize}
			
		\subsubsection*{15.05.2014 - Spotkanie z klientem}
		
			Po zapoznaniu się z dokumentami teoretycznymi dotyczącymi SPH, które otrzymaliśmy od klienta, spotkaliśmy się z klientem aby wyjaśnić powstałe dotychczas wątpliwości oraz przedstawić nasze przemyślenia dotyczące realizacji pracy i dotychczasowe rezultaty.
			
			\begin{itemize}
			
				\item zaproponowane sposoby wyznaczania sąsiadów
				
					\begin{itemize}
					
						\item naiwne sprawdzanie odległości wszystkich pozostałych cząstek aż do znalezienia wymaganej liczby sąsiadów (rozwiązanie najmniej optymalne, jednakże łatwe do implementacji i przydatne w fazie prototypowania oraz do porównywania z lepszymi metodami)
						\item podzielenie pudła obliczeniowego na mniejsze sześciany o takiej samej długości boku, mapowanie położenia cząstki na odpowiedni sześcian, a następnie szukanie sąsiadów jedynie w sześcianach bezpośrednio przyległych do rogu sześcianu, w którym znajduje się cząstka (8 sześcianów)
						\item podzielenie cząstek przy pomocy struktury danych oct-tree, podobne rozwiązanie do poprzedniego, jednakże umożliwia podanie oczekiwanej ilości cząstek w jednym węźle struktury oraz pozwala na symulację płynu poza pudłem obliczeniowym
					
					\end{itemize}
					
				\item prezentacja dotychczasowych rezultatów:
				
					\begin{itemize}
					
						\item wstępny, działający moduł graficzny aplikacji, umożliwiający wszystkie transformacje kamery w przestrzeni trójwymiarowej
						\item możliwość ładowania modeli i umieszczania ich w przestrzeni, przydatne m.in. do stworzenia otoczenia w jakim będzie przeprowadzana symulacja (podłoże, wizualizacja pudła obliczeniowego)
						\item prototyp modułu obliczeniowego, zawierający inicjalizację OpenCL oraz demonstracyjne przekształcenia położeń cząstek
					
					\end{itemize}
			
			\end{itemize}
		
		\subsubsection*{08.10.2014 - Spotkanie z klientem}
		
		\subsubsection*{02.12.2014 - Spotkanie z klientem}
			
		\subsubsection*{05.01.2015 - Spotkanie z prowadzącym pracownię projektową}

\section{Podsumowanie}

	\subsection{Realizacja założeń i dalszy rozwój}
	
		Udało się stworzyć aplikację będącą PoC (Proof of Concept, ang. dowód koncepcji) badanego problemu. Aplikacja ta przeprowadza symulację zachowania płynu przy użyciu metody SPH zaimplementowanej jako seria funkcji OpenCL. Wyniki przeprowadzanej symulacji wizualizowane są w czasie rzeczywistym przy użyciu OpenGL. Płynność uzyskanej animacji zależy od ilości symulowanych cząstek, jednakże w porównaniu z innymi symulacjami, których wyniki dostępne są w internecie, spełnia nasze oczekiwania.
		
		Poniższe elementy nie zostały zrealizowane w pełni bądź pominięte, głównie z powodu braku czasu. Nie są to elementy niezbędne do poprawnego działania projektu, jednakże ich realizacja bardzo pomogłaby w dalszym rozwoju stworzonego oprogramowania.
		
		\begin{itemize}
		
			\item Stworzenie przejrzystego interfejsu użytkownika dostarczającego informacje potrzebne do debugowania kodu. Okazało się, że poprawna implementacja wszystkich elementów logiki aplikacji jest niewystarczająca do poprawnego działania symulacji, kluczowym jest także dobór odpowiednich stałych parametrów równań. Pomimo iż udało nam się dobrać przykładową poprawną parametryzację, zmiana dowolnej stałej wymaga dostosowania wszystkich pozostałych, co bez znajomości informacji takich jak np. maksymalne/średnie/minimalne ciśnienie występujące w chwili obecnej w układzie jest niezwykle uciążliwe i trudne. Kluczowym jest więc stworzenie modułu odpowiedzialnego za wyciąganie takich danych z pamięci karty graficznej, a następnie wyświetlanie ich w postaci tekstu w oknie graficznym aplikacji. Do osiągnięcia tego celu zaleca się wymianę biblioteki odpowiedzialnej za zarządzanie oknem (w chwili obecnej GLFW) na bibliotekę o wyższym poziomie abstrakcji (np. SDL, SFML), co umożliwiłoby zarówno zachowanie funkcjonalności zarządzania oknem graficznym, jak i łatwą integrację wyświetlania tekstu.
			
			\item Podpięcie odpowiednich zdarzeń związanych z wprowadzaniem wejścia z klawiatury do logiki odpowiedzialnej za zmianę stałych parametrów równań. Umożliwiłoby to zmianę stałych w trakcie wykonywania symulacji i jednoczesną obserwację ich poprawności. Dzięki temu możliwe byłoby dużo łatwiejsze dobranie poprawnych parametrów poprzez bezpośrednie badanie ich wpływu na zachowanie symulowanego płynu.
			
			\item Możliwa jest także wymiana sposobu liczenia składowej lepkości przy obliczaniu przyspieszenia cząstki. Obecny sposób może nie dawać najbliższych rzeczywistości rezultatów wizualnych.
			
			\item W chwili obecnej możliwe jest przeprowadzenie symulacji w prostopadłościennym pudle obliczeniowym poprzez umieszczenie w nim prostopadłościennej ściany równo oddalonych od siebie cząstek płynu. Następnie pod wpływem grawitacji i sił oddziaływań między nimi zaczynają się one przemieszczać w obrębie pudła. Aby zaobserwować bardziej skomplikowane zjawiska i właściwości obserwowanego układu należałoby zaimplementować moduł odpowiedzialny za tworzenie i zarządzanie prostymi bryłami kolizyjnymi. Bryły pełniłyby wewnątrz pudła tą samą funkcję co ściany pudła, jednakże składając je w bardziej skomplikowane kształty można by zaobserwować zjawiska takie jak opływanie przeszkody z dwóch stron przez cząstki płynu oraz wydostawanie się cząstek płynu z naczynia przez otwór.
			
			\item Możliwa jest dalsza optymalizacja obliczeń w celu uzyskania mniejszych wymagań ze strony sprzętu, co pozwoliłoby na zwiększenie dopuszczalnej ilości cząstek biorących udział w symulacji lub zwiększyłoby zakres sprzętu, na którym można przeprowadzić symulację. M.in. należałoby rozważyć które bloki instrukcji wykonywanych sekwencyjnie w kernelach OpenCL można zastąpić przez wykonanie ich za pomocą kerneli lokalnych.
		
		\end{itemize}
		
	\subsection{Narzędzia wspierające proces wytwarzania oprogramowania}
	
		Do zarządzania procesem wytwarzania oprogramowania wykorzystywane były głównie narzędzia dostępne w obrębie serwisu hostingowego GitHub:
		
		\begin{itemize}
		
			\item System kontroli wersji Git, służył do synchronizacji pracy pomiędzy lokalnymi środowiskami autorów. Pozwalał na zachowywanie stabilnych wersji oprogramowania i powracanie do nich w razie potrzeby. Był też jedną z formy komunikacji pomiędzy autorami dzięki możliwości nazywania pakietów wprowadzanych zmian (commit'ów).
			
			\item Na pewnym etapie do zarządzania zadaniami i planowania kolejnych kierunków rozwoju projektu używana była webowa aplikacja Trello, która umożliwia rozwijanie m.in. oprogramowania przy wykorzystaniu metody kanban. Dzięki niej możliwy był podział wszystkich zadań na trzy grupy: nowe, w trakcie realizacji oraz wykonane.
		
		\end{itemize}
		
		W trakcie realizacji projektu szczególnie przydatne okazały się następujące techniki:
		
		\begin{itemize}

			\item Rubber duck debugging - rolę gumowej kaczki najczęściej pełniła druga osoba zajmująca się projektem, przedstawianie jej krok po kroku segmentów debugowanego kodu bardzo często pomagało autorowi uświadomić sobie naturę błędu oraz metoda ta miała przewagę nad jej klasyczną wersją, ponieważ osoba pełniąca rolę gumowej kaczki mogła zadawać kluczowe pytania pomagające zrozumieć jej działanie kodu, co często naprowadzało autora na trop błędu.
			
			\item Pair programming - metoda szczególnie pasująca do realizowanego projektu, gdyż był on rozwijany w zespole dwuosobowym.
			
			\item Wymieniona wcześniej metoda kanban
		
		\end{itemize}
		
		Poza tym w celu komunikacji pomiędzy członkami zespołu wykorzystywane były:
		
		\begin{itemize}
		
			\item Wiadomości e-mail do kontaktowania się z managerem, klientem oraz pomiędzy członkami zespołu. Umożliwiały też wymianę dokumentów naukowych na podstawie których tworzony był projekt oraz wszelkich kluczowych dokumentów znalezionych w internecie, dotyczących aspektów technicznych.
			
			\item Komunikatory internetowe (np. Google Hangouts), były głównie używane w trakcie zdalnej pracy członków zespołu do rozmowy o obecnie realizowanych zadaniach.
		
		\end{itemize}
		
	\subsection{Podział prac}
	
		\subsubsection*{Dawid Romanowski} 
		
		\begin{itemize}
		
			\item w ramach dokumentacji technicznej
			
				\begin{itemize}
				
					\item Opis OpenCL
					\item Struktura projektu
					\item Sposób implementacji metody SPH
					\item Sposób implementacji wizualizacji metody SPH
					\item Opis wykorzystanych zewnętrznych bibliotek
				
				\end{itemize}
			
			\item w ramach dokumentacji procesowej
			
				\begin{itemize}
				
					\item Pomoc przy formułowaniu celów pracy
					\item Zapisywanie minutek w trakcie spotkań
				
				\end{itemize}
			
			\item w ramach dokumentacji użytkownika
			
				\begin{itemize}
				
					\item Opis metody SPH - od wstępu do opisu funkcji wygładzających włącznie
					
					\item GPGPU i OpenCL
					
					\item Korzystanie z programu wraz z zrzutami ekranu
				
				\end{itemize}
			
			\item zaplanowanie oraz implementacja finalnej architektury systemu
			
			\item implementacja kerneli związanych z wyszukiwaniem sąsiadów
			
			\item zarządzanie plikami CMake w celu zapewnienia poprawnej kompilacji i linkowania aplikacji w systemach z rodziny debian (Linux Mint, Ubuntu).
		
		\end{itemize}
		
	\subsubsection*{Wojciech Czarny} 
		
		\begin{itemize}
			
			\item w ramach dokumentacji procesowej
			
				\begin{itemize}
				
					\item Wizja projektu, cel projektu, opisy problemu i użytkowników, wymagania.
					
					\item Dokumentacja przebiegu pracy, kamieni milowych, minutek.
					
					\item Sformułowanie podsumowania, określenie stopnia realizacji założeń i dalszych kierunków rozwoju, udokumentowanie użytych narzędzi oraz podziału prac
				
				\end{itemize}
			
			\item w ramach dokumentacji użytkownika
			
				\begin{itemize}
				
					\item Opis metody SPH - część wzorów użytkowych, 
					
					\item GPGPU i OpenCL
				
				\end{itemize}
			
			\item stworzenie prototypu modułu graficznego i obliczeniowego
			
			\item wsparcie przy implementacji finalnej architektury systemu
			
			\item implementacja kerneli związanych z zarządzaniem danymi oraz kerneli realizujących obliczenia metody SPH
					
			\item zarządzanie plikami CMake w celu zapewnienia poprawnej kompilacji i linkowania aplikacji w systemie Windows 8.1
		
		\end{itemize}
		

\end{document}